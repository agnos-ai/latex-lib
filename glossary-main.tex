%%
%% The main glossary, use this for any terms that are NOT:
%%
%% - Concepts (use glossary-concepts.tex for that)
%% - Ontologies (use glossary-ontologies.tex for that)
%%
%% Do also not store any customer specific terms here, use
%% glossary-<customer-code>.tex for that.
%%

\newglossaryentry{master}{
    type=\glsdefaulttype,
    name={Master-data},
    description={Business critical data about parties, places and things}
}

\newglossaryentry{non-master}{
    type=\glsdefaulttype,
    name={Non-master data},
    description={Transactional data}
}

\newglossaryentry{temporal}{
    type=\glsdefaulttype,
    name={Temporal},
    description={Relating to Time}
}

\newglossaryentry{edgc}{
    type=\glsdefaulttype,
    name={Enterprise Data Governance Council},
    description={An enterprise governing body responsible for the data governance strategy,
    setting the organization-wide data policies and standards, and communicating them
    to enforce the data governance program}
}

\newglossaryentry{data-stewards}{
    type=\glsdefaulttype,
    name={Data Stewards},
    description={One who establishes data requirements and assesses the
    quality of the data in the data stores}
}

\newglossaryentry{bitemporality}{
    name={Bi-temporality},
    type=\glsdefaulttype,
    name={EKG/Monitoring},
    parent={ekg},
    description={..todo..}
}

\newglossaryentry{bitemporality}{name={Bi-temporality},%
    type=\glsdefaulttype,%
    description={Bi-temporality is a feature of a system (or an Ontology) that allows for the recording of timestamps along two time lines:
    the time when the event was happening in the real world and the time when the event was recorded. See also "multi-temporality".}
}

\newglossaryentry{system-architecture}{
    type=\glsdefaulttype,
    name={system architecture},
    description={
        A system architecture is the conceptual model that defines the structure, behavior, and more views of a system.
    }
}

\newglossaryentry{object}{
    name={Object},%
    type=\glsdefaulttype,%
    description={A knowledge resource, that is intended to be referenceable and shared.
    It does not include structured values (\textit{blank nodes}) used in only one place.}
}

\newglossaryentry{uri}{
    name={URI},%
    type=\glsdefaulttype,%
    first="Uniform Resource Identifier",
    description={A Uniform Resource Identifier (URI) is a compact sequence of
    characters that identifies an abstract or physical resource. See https://www.ietf.org/rfc/rfc3986.txt.}
}

\newglossaryentry{iri}{
    name={IRI},%
    type=\glsdefaulttype,%
    first="Internationalised Resource Identifier (IRI)",
    description={
        The \gls{w3c} \gls{rdf} 1.1 standard, the most fundamental standard that underpins any \gls{ekg},
        The IRI standard is a superset of the older \gls{uri} standard.
    }
}

\newglossaryentry{ekg:iri}{
    name={EKG/IRI},%
    text={EKG/IRI\glsadd{iri}},
    type=\glsdefaulttype,%
    first="Enterprise Knowledge Graph IRI (EKG/IRI)",
    description={
        An \gls{iri} that forms the identity of an object in the \gls{ekg}.
        Any given object in an \gls{ekg} has an EKG/IRI for which special rules are defined by the \gls{ekgf}.
        Not to be confused by \glspl{canonical-identifier}.
    }
}

\newglossaryentry{canonical-identifier}{
    name={Canonical Identifier},%
    type=\glsdefaulttype,%
    description={A permanent identifier for an object, distinguished within an \gls{ekg}.}
}

\newglossaryentry{property}{
    name={Canonical Identifier},
    type=\glsdefaulttype,%
    description={An identified and fully-defined linkage between an object and values.}
}

\newglossaryentry{ontology}{
    name={Ontology},
    type=\glsdefaulttype,%
    description={A machine-readable, managed and identified set of properties and related definitions.}
}

\newglossaryentry{data-point}{
    name={Data Point},
    type=\glsdefaulttype,%
    description={A fact representing the value(s) of a property for an object in some context.
    Each value may be a simple data (e.g. a number, string or date) or another object.
    The combination of object, property and value is a \textit{triple}.}
}

\newglossaryentry{dataset}{
    name={Dataset},
    type=\glsdefaulttype,%
    description={A collection of data, published or curated by a single agent,
    and available for access or download in one or more serializations or formats.}
}




