%%
%% The main glossary, use this for any terms that are NOT:
%%
%% - Concepts (use glossary-concepts.tex for that)
%% - Ontologies (use glossary-ontologies.tex for that)
%%
%% Do also not store any customer specific terms here, use
%% glossary-<customer-code>.tex for that.
%%

\newglossaryentry{master}{
    type=\glsdefaulttype,
    name={Master-data},
    description={Business critical data about parties, places and things}
}

\newglossaryentry{non-master}{
    type=\glsdefaulttype,
    name={Non-master data},
    description={Transactional data}
}

\newglossaryentry{temporal}{
    type=\glsdefaulttype,
    name={Temporal},
    description={Relating to Time}
}

\newglossaryentry{edgc}{
    type=\glsdefaulttype,
    name={Enterprise Data Governance Council},
    description={An enterprise governing body responsible for the data governance strategy,
    setting the organization-wide data policies and standards, and communicating them
    to enforce the data governance program}
}

\newglossaryentry{data-stewards}{
    type=\glsdefaulttype,
    name={Data Stewards},
    description={One who establishes data requirements and assesses the
    quality of the data in the data stores}
}

\newglossaryentry{ekg:platform}{
    type=\glsdefaulttype,
    name={EKG/Platform},
    parent={ekg},
    description={a system architecture concept that stands for the layer of software services that provide
    and serve the \gls{ekg} to end-users and other systems.
    }
}

\newglossaryentry{ekg:storage}{
    type=\glsdefaulttype,
    name={EKG/Storage},
    text={EKG/Storage},
    parent={ekg},
    description={a system architecture concept that stands for the layer of data storage services such as a "Triple store" or an
    "Object Store" that serve the various other layers of a typical \gls{ekg} operation such as the \gls{ekg:platform} and the \glspl{ekg:dataops:pipeline}. }
}

\newglossaryentry{ekg:dataops}{
    type=\glsdefaulttype,
    name={EKG/DataOps},
    text={EKG/DataOps},
    parent={ekg},
    description={..todo..}
}

\newglossaryentry{ekg:dataops:environment}{
    type=\glsdefaulttype,
    name={Environment},
    text={EKG/DataOps Environment},
    parent={ekg:dataops},
    description={..todo..}
}

\newglossaryentry{ekg:dataops:pipeline}{
    type=\glsdefaulttype,
    name={Pipeline},
    text={EKG/DataOps Pipeline},
    parent={ekg:dataops},
    description={a pipeline in the \gls{ekg:dataops:environment} is a series of programs, called "steps",
    that are run in sequence where the first step captures data in any given format from a given source 
    and the last step produces an output file in any given format.}
}

\newglossaryentry{ekg:monitoring}{
    type=\glsdefaulttype,
    name={EKG/Monitoring},
    parent={ekg},
    description={..todo..}
}

\newglossaryentry{bitemporality}{name={Bi-temporality},%
    type=\glsdefaulttype,%
    description={Bi-temporality is a feature of a system (or an Ontology) that allows for the recording of timestamps along two time lines:
    the time when the event was happening in the real world and the time when the event was recorded. See also "multi-temporality".}
}

