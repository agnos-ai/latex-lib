%%
%% The main glossary, use this for any terms that are NOT:
%%
%% - Concepts (use glossary-concepts.tex for that)
%% - Ontologies (use glossary-ontologies.tex for that)
%%
%% Do also not store any customer specific terms here, use
%% glossary-<customer-code>.tex for that.
%%

\newglossaryentry{master}{
    type=\glsdefaulttype,
    name={Master-data},
    description={Business critical data about parties, places and things}
}

\newglossaryentry{non-master}{
    type=\glsdefaulttype,
    name={Non-master data},
    description={Transactional data}
}

\newglossaryentry{cde}{
    type=\glsdefaulttype,
    name={critical data element},
    text={Critical data element},
    first={Critical Data Element (CDE)},
    plural={Critical data elements},
    firstplural={Critical Data Elements (CDEs)},
    description={%
        this concept is defined in many different ways.
        One way is to say that "critical data elements are those data elements that are critical to success in a
        specific business area".
        This could be a regulatory area such as BCBS 239 or the enterprise as a whole or any other use case area.
        In an \gls{ekg} context, especially at the higher levels of \gls{ekgmm} maturity,
        \myuline{all data is critical} since the criticality of data is always dependent on the context.
        For any given data point, there is a raison d'être in a given context.
        See also \chapterref{sec:ekg-mm-b-4-5}{Critical Data Elements}{\glsfmtshort{ekgmm}}.
    }
}

\newglossaryentry{temporal}{
    type=\glsdefaulttype,
    name={Temporal},
    description={Relating to Time}
}

\newglossaryentry{edgc}{
    type=\glsdefaulttype,
    name={Enterprise Data Governance Council},
    description={%
        is an enterprise governing body responsible for the data governance strategy,
        setting the organization-wide data policies and standards, and communicating them
        to enforce the data governance program
    }
}

\newglossaryentry{data-steward}{
    type=\glsdefaulttype,
    name={Data Steward},
    plural={Data Stewards},
    description={%
        One who establishes data requirements and assesses the
        quality of the data in the data stores
    }
}

\newglossaryentry{bitemporality}{
    name={bi-temporality},
    type=\glsdefaulttype,
    text={Bi-temporality},
    description={%
        a feature of a system (or an Ontology) that allows for the recording of timestamps along two timelines:
        the time when the event was happening in the real world and the time when the event was recorded.
        See also "multi-temporality".
    }
}

\newglossaryentry{system-architecture}{
    type=\glsdefaulttype,
    name={system architecture},
    description={%
        the conceptual model that defines the structure, behavior, and more views of a system.
    }
}

\newglossaryentry{organizational-accountability} {
    type=\glsdefaulttype,
    name={organizational accountability},
    plural={organizational accountabilities},
    description={%
        is about defining the company's mission, values, and goals,
        as well as everyone's role in working toward them.
        It's about holding employees and executives responsible for accomplishing these goals, completing assignments,
        and making decisions that deliver on these expectations.
    }
}

\newglossaryentry{technology-stack} {
    type=\glsdefaulttype,
    name={technology stack},
    plural={technology stacks},
    description={%
        ...
    }
}

\newglossaryentry{data-incongruence} {
    type=\glsdefaulttype,
    name={data incongruence},
    plural={data incongruences},
    description={%
        ...
    }
}

\newglossaryentry{object}{
    name={Object},%
    type=\glsdefaulttype,%
    description={A knowledge resource, that is intended to be referenceable and shared.
    It does not include structured values (\textit{blank nodes}) used in only one place.}
}

\newglossaryentry{uri}{
    name={URI},%
    type=\glsdefaulttype,%
    first="Uniform Resource Identifier (URI)",
    firstplural="Uniform Resource Identifiers (URIs)",
    description={%
        a Uniform Resource Identifier (URI) is a compact sequence of characters that identifies an abstract or
        physical resource.
        See \url{https://www.ietf.org/rfc/rfc3986.txt} and \gls{iri}.
    }
}

\newglossaryentry{iri}{
    name={IRI},%
    type=\glsdefaulttype,%
    first="Internationalised Resource Identifier (IRI)",
    firstplural="Internationalised Resource Identifiers (IRIs)",
    description={%
        an Internationalized Resource Identifier is defined similarly to a \glsxtrshort{uri},
        but the character set is extended to the Universal Coded Character Set.
        Therefore, it can contain any Latin and non-Latin characters except the reserved characters.
        Instead of extending the definition of \glsxtrshort{uri},
        the term IRI was introduced to allow for a clear distinction and avoid incompatibilities.
        IRIs are meant to replace URIs in identifying resources in situations where the Universal Coded Character Set
        is supported.
        By definition, every URI is an IRI.
        Furthermore, there is a defined surjective mapping of IRIs to URIs:
        Every IRI can be mapped to exactly one URI, but different IRIs might map to the same URI.
        Therefore, the conversion back from a URI to an IRI may not produce the original IRI.
        The IRI standard is a superset of the older \gls{uri} standard (IRI ⊃ URI).
        See also \gls{ekg:iri}.
    }
}

\newglossaryentry{linked-data}{
    name={linked data},% how it appears in the glossary
    type=\glsdefaulttype,%
    text={Linked Data},% how it appears in the doc
    description={%
        the collection of interrelated datasets on the Web.
        Linked Data is a concept defined by Tim Berners-Lee and the \gls{w3c} that is part of what they call
        "the Semantic Web" or "the Web of Data".
        Tim Berners-Lee started the idea in 2006 by defining 4 simple rules:
        \url{https://www.w3.org/DesignIssues/LinkedData.html}.
    }
}

\newglossaryentry{ekg:iri}{
    name={EKG/IRI},%
    text={EKG/IRI\glsadd{iri}},
    type=\glsdefaulttype,%
    first="Enterprise Knowledge Graph IRI (EKG/IRI)",
    description={%
        an \gls{iri} that forms the identity of an object in the \gls{ekg}.
        Any given object in an \gls{ekg} has an EKG/IRI for which special rules are defined by the \gls{ekgf}.
        Not to be confused by \glspl{canonical-identifier}.
    }
}

\newglossaryentry{canonical-identifier}{
    name={Canonical Identifier},%
    type=\glsdefaulttype,%
    description={A permanent identifier for an object, distinguished within an \gls{ekg}.}
}

\newglossaryentry{property}{
    name={property},
    type=\glsdefaulttype,%
    text={property},
    description={%
        an identified and fully-defined linkage between an object and values.
        Or in \gls{rdf} terminology, a relation between subject resources and object resources.
        See \url{https://www.w3.org/TR/rdf-schema/\#ch_property}.
    }
}

\newglossaryentry{ontology}{
    name={ontology},% how it appears in the glossary
    type=\glsdefaulttype,%
    text={Ontology},% how it appears in the text
    description={
        a machine-readable, managed and identified set of properties and related definitions.
    }
}

\newglossaryentry{data-point}{
    name={data point},% how it appears in the glossary
    type=\glsdefaulttype,%
    text={Data Point},% how it appears in the text
    description={%
        a fact representing the value(s) of a property for an object in some context.
        Each value may be a simple data (e.g. a number, string or date) or another object.
        The combination of object, property and value is a \textit{triple}.
    }
}

\newglossaryentry{dataset}{
    name={dataset},
    type=\glsdefaulttype,%
    text={Dataset},
    description={%
        a collection of data, published or curated by a single agent,
        and available for access or download in one or more serializations or formats.
    }
}

\newglossaryentry{operating-model} {
    type=\glsdefaulttype,
    name={operating model},
    plural={operating models},
    description={%
        is both an abstract and visual representation of how an organization delivers value to its customers or
        beneficiaries as well as how an organization actually runs itself.
    }
}

\newglossaryentry{lighthouse-project} {
    type=\glsdefaulttype,
    name={lighthouse project},
    text={Lighthouse project},
    plural={Lighthouse projects},
    description={%
        Lighthouse projects simply focus on implementation, fast delivery and creating a positive culture for
        digital transformation.
        A lighthouse project is a short-term, well defined, measurable project that serves as a model\,---\,or
        a “lighthouse”\,---\,for other similar projects within the broader digital transformation initiative.
        See \href{https://www.contino.io/insights/why-lighthouse-projects-not-powerpoints-will-unlock-your-transformation-value}
        {Why Lighthouse Projects, Not PowerPoints, Will Unlock Your Transformation Value}
    }
}

\newglossaryentry{low-code} {
    type=\glsdefaulttype,
    name={low-code},
    text={Low-Code},
    description={%
        is a software development approach that requires little to no coding in order to build applications and processes.
        In an \gls{ekg} context, low-code\,---\,or actually it's more like "no-code"\,---\,development is achieved by
        specifying business and user requirements at such a level of detail, with or without the use of
        visual user interfaces for developers, that at some point smart "model executing services" can provide a running
        production-level use case.
    }
}
